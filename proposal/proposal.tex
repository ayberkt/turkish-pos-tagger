\documentclass{article}
\usepackage{natbib}
\usepackage{hyperref}

\newcommand{\hmmURL}{https://hackage.haskell.org/package/hmm-0.2.1.1/docs/Data-HMM.html}

\title{\bf An HMM-based part-of-speech tagger for Turkish}
\author{Ayberk Tosun\\\texttt{tosun2@illinois.edu}}
\date{}

\begin{document}
\maketitle

\section{Motivation}
\label{sec:motivation}
My main motivation for undertaking this project is to understand the main
principles underlying POS-tagging; due to time and resource constraints, it is
not possible for me to build a state-of-the-art POS-tagger for
Turkish. Nevertheless, I expect my POS-tagger
to work with a decent accuracy that is enough to be a starting point for
further development. Moreover, \citet{Korkut2015} began the development of a
Turkish NLP framework named ``Guguk'' implemented in the programming language Haskell. I intend
to incorporate this project to the Guguk project, hence contributing towards
the existence of a Turkish NLP-framework in Haskell.

\section{Related Work}
\label{sec:related_work}

% Briefly survey the most salient prior work that relates to the problem you wish
% to solve. This section should cite relevant sources. This section should be at
% least one to two paragraphs in length.
The most important advancement in the computational linguistics of Turkish is
the NLP framework ``Zemberek'' by \citet{akin2007zemberek} that has been
immensely useful for both research and applications of Turkish NLP. Although it started
out as an NLP tool for the Turkish language, Zemberek moved on to become a
a tool for processing all Turkic languages. Zemberek

The widely cited \citet{oflazer1994tagging} outlines the implementation of a
tool for both morpheme glossing and POS tagging, that has ``98-99 \%'' accuracy. This
tool is based on a morphological specification of Turkish and makes use of a
morphological processor called ``PC-Kimmo'' \citep{antworth1991pc}.

Most of the literature relating to the computational processing of Turkish
pertain primarily to Turkish morphology---compared to the complexity of Turkish
morphology, its syntax seems to be a less interesting problem.
\section{Proposal}

I will be implementing a part-of-speech tagger for the Turkish language. As
mentioned in section~\ref{sec:related_work} I will be mostly focusing on parts-of-speech
of words\footnote{I base my defition of a ``word'' on the orthographic space in
  most of the cases. Some of the exceptions to this are Turkish reduplication
  that have the POS ``Dup''} and will not be accounting for morphology. There are problems with this
approach but this project aims to be a starting point for further research.

For programming the POS tagger I will be using Haskell, a statically-typed and purely
functional programming language. I will be using the
\href{\hmmURL}{\texttt{Data.HMM}} package for
Haskell. The data, that is probably the most important component of this
project, comes from the METU-Sabanc\i treebank built by
\citet{oflazer2003building}. I will be stripping the POS tag for each word from
the corpus and will use them to train HMM's.

\bibliographystyle{apalike}
\bibliography{your_bibliography}

\end{document}
